\documentclass{article}
\usepackage[utf8]{inputenc}

\title{Pin}
\author{Jonas Gonçalves}
\date{September 2019}

\begin{document}

\maketitle
\section{Introduction}
\qquad Short for \textbf{personal identification number, PIN} is a numeric of alpha-numeric password used in the process of authenticating a user accessing a system. It's often used with automated bank teller machines, telephone calling cards, and accessing Wireless networks

The personal identification number has been the key to flourishing the exchange of private data between different data-processing centers in computer networks. PINs can also be used to authenticate banking systems with carholders, governments with citizens, enterprises with employees, and with users, among other uses.
\begin{itemize}
    \item[1.] A four-digit PIN is selected. What's the probability that there are no repeated digits?
    
    \item[2.] What's the probability that a six-digit PIN has at least one repeated digit?
\end{itemize}

\section{Solution 1}

\quad There are 10 possible values for each digit of the PIN (0,1,2,...,9).

There are a total of \(10^4\) possible PINs.
\[P_T = 10 \times 10 \times 10 \times 10 = 10^4\]

To have no repeated digit, all four digit would have to be different, which is selecting without replacement. We could either compute \(10 \times 9 \times 8 \times 7\) , or notice that this is the same permutation as \[7 \times 6 \times 5 \times 4 \times 3 \times 2 \times 1 = 7!\]
\[P_4 = 7! = 5040\]

The probability of no repeated digits is the number of 4 digit PINs with no repeated digit divided by the total of 4 digit PINs.
\[Pr = \frac{Pr_4}{Pr_T} = \frac{5040}{10000} = 0.504\]

\section{Solution 2}
The probability of having at least one repeated digit is the 1 minus the probability of not having any repeated digits.
\[Pr = 1 - \frac{P_6}{P_T}\]
\[P_T = 10^6\]
\[P_6 = 10 \times 9 \times 8 \times 7 \times 6 \times 5 = \frac{10!}{4!} = 151200\]

\[P = 1 - \frac{151200}{10^6} = 1 - 0.1512 = 0.8488\]
\end{document}

\documentclass[a4paper, 12pt]{article}
\usepackage[portuges]{babel}
\usepackage[utf8x]{inputenc}
\usepackage[T1]{fontenc}
\usepackage[a4paper,top=3cm,bottom=2cm,left=3cm,right=3cm,marginparwidth=1.75cm]{geometry} 
\usepackage{float}
\usepackage{graphicx}
\usepackage{caption}
\usepackage{wrapfig}

\usepackage[colorlinks=False]{hyperref}

\usepackage{amsmath, amsfonts, amssymb} 
\usepackage{color} 
\usepackage{enumitem}
\usepackage{minted} 


\begin{document}

\section*{Questão 7 - A Disputa das Chapas}
\large {John é um jovem muito interessado nos movimentos estudantis e logo no seu primeiro ano já decidiu em participar de alguma chapa para participar mais ativamente. Na sua faculdade existem cinco chapas principais A, B, C, D e E, com intenções de voto 0.40, 0.30, 0.15, 0.10 e 0.05, respectivamente. John muito ansioso em meio ao ambiente rapidamente se aproximou de diferentes chapas e se inscreveu. 
\newline
Contudo para evitar esse tipo de situação foram criadas duas comissões para fiscalizar as eleições das chapas. A comissão H tem uma chance de 0.10 de detectar que um aluno participa de mais de uma chapa e caso isso ocorra o aluno será punido não podendo se inscrever em outra chapa. Já a comissão J atua sobre a investigação da comissão H e tem uma chance de 0.10 de punir as chapas que o mesmo aluno está inscrito, não as permitindo de participar das eleições.
\newline
No caso da punição da chapa, os votos das chapas elimidas serão repassados para uma chapa que não foi punida, sendo que a probabilidade de receber esses votos é a mesma para todas as chapas não punidas. Levando em conta que o unico aluno que participa de mais de uma chapa é John, pede-se}
\newline
\newline
\textbf{A.}  Sabendo que John pode ter se inscrito em 2, 3, 4 ou 5 das chapas existentes. De quais maneiras a chapa C pode vencer as eleições e quais suas respectivas probabilidades?
\newline
\newline
\textbf{B.} Refaça o item \textbf{A} agora para a chapa D e a E.
\newline
\newline
\textbf{C.} Caso houvesse mais um estudante além de John, que também se inscrevesse em 2, 3, 4 ou 5 das chapas existentes. Quais as maneiras que John poderia ganhar as eleições com uma de suas chapas e suas respectivas probabilidades?




















\end{document}
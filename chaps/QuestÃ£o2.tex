\documentclass[a4paper, 12pt]{article}
\usepackage[portuges]{babel}
\usepackage[utf8x]{inputenc}
\usepackage[T1]{fontenc}
\usepackage[a4paper,top=3cm,bottom=2cm,left=3cm,right=3cm,marginparwidth=1.75cm]{geometry} 
\usepackage{float}
\usepackage{graphicx}
\usepackage{caption}
\usepackage{wrapfig}

\usepackage[colorlinks=False]{hyperref}

\usepackage{amsmath, amsfonts, amssymb} 
\usepackage{color} 
\usepackage{enumitem}
\usepackage{minted} 


\begin{document}

\section*{Questão 2 - Genética de Populações}
\large {Uma determinada população de ratos tem a cor dos seus pelos determinada por dois loci gênicos, sendo que um deles está relacionado à cor e o outro à produção do pigmento, e a produção de pigmento é dominante em relação ao albinismo. Sabendo-se que o cruzamento de um rato preto homozigoto com um rato cinza também homozigoto resultada em um rato marrom, pede-se:}
\newline
\newline
\textbf{A.}  Escreva os genótipos e os respectivos fenótipos para os ratos.
\newline
\newline
\textbf{B.} Partindo de uma população de ratos em que a porcentagem de ratos albinos é 0,01 e que a frequência do gene, relacionado a cor, presente tanto nos ratos pretos quanto nos marrons, é 0,2. Obtenha a frequência de cada fenótipo.
\newline
\newline
\textbf{C.}  Caso houvesse uma mutação no locus associado a produção de pigmento e os embriões albinos fossem inviabilizados antes do nascimento (gene letal), haveria uma mudança no resultado do item \textbf{B}? Se sim, recalcule-o.
\newline
\newline
\textbf{D.} Generalize os itens \textbf{B} e \textbf{C}.



















\end{document}

\documentclass[a4paper, 12pt]{article}
\usepackage[portuges]{babel}
\usepackage[utf8x]{inputenc}
\usepackage[T1]{fontenc}
\usepackage[a4paper,top=3cm,bottom=2cm,left=3cm,right=3cm,marginparwidth=1.75cm]{geometry} 
\usepackage{float}
\usepackage{graphicx}
\usepackage{caption}
\usepackage{wrapfig}

\usepackage[colorlinks=False]{hyperref}

\usepackage{amsmath, amsfonts, amssymb} 
\usepackage{color} 
\usepackage{enumitem}
\usepackage{minted} 


\begin{document}

\section*{Questão 5 - Amigo Secreto}
\large {Um grupo de 10 amigos, após um longo e exaustivo ano, resolveu fazer um amigo secreto numa confraternização de final de ano da empresa em que trabalhavam. Entretanto, durante os sorteios dos nomes, eles se depararam com um problema, algumas pessoas tiravam o próprio nome e eles tinham que refazer todo o processo. Assim, partindo disso responda:}
\newline
\newline
\textbf{A.}  De quantas maneiras distintas podem ser organizadas as pessoas nesse amigo secreto, sem que uma pessoa tire ela mesma?
\newline
\newline
\textbf{B.} Levando em conta que quando uma pessoa se sorteia seja realizado um novo sorteio automaticamente. Qual a probabilidade do sorteio ser bem sucedido na primeira tentativa?
\newline
\newline
\textbf{C.}  Com quantos sorteios sucessivos a probabilidade de um bem sucedido é maior que 0.5?
\newline
\newline
\textbf{D.} Generalize o item \textbf{A} para N amigos.



















\end{document}
\documentclass{article}
\usepackage[utf8]{inputenc}

\title{A brief historical overview of probability}
\author{Gabriel Aguiar}
\date{August 2019}

\begin{document}

\maketitle

\section{Introduction}

\hfill

Tudo começou no século XVI, quando o italiano Gerolamo Cardano colocou no papel pela primeira vez o arcabouço para a análise de situações que envolvem o acaso.

Cardano não era do tipo intelectual europeu, mas sim um sujeito que acreditava no destino, na sorte e no vislumbre do futuro através do alinhamento de planetas e estrelas. Sua marca na Teoria das Probabilidades se deu em meio à sua habilidade com jogos de azar. Contudo, o desenvolvimento de ferramentas matemáticas para prever resultados em um jogo não ocorreu através da racionalidade, mas sim da intuição.

O italiano, em meio a sérios problemas pessoais, viu nos jogos de apostas uma oportunidade de conseguir dinheiro. Desta maneira, adquirindo uma apurada compreensão da possibilidade de vencer frente a situações distintas, Gerolamo brilhou.

Com isso, Cardano produziu "O livro dos jogos de azar", obra que representa a primeira tentativa humana de compreender a essência da incerteza. Nela, fundamenta-se o que hoje conhecemos como "Lei do Espaço Amostral".

Na época em que viveu, a aritmética e a álgebra ainda se encontravam em estado primitivo. Tal fato torna seu trabalho ainda mais notável.

Galileo Galilei também foi um dos pioneiros a embarcar na análise de situações regidas pelo acaso. Ele chegou a escrever um breve artigo sobre os jogos de azar utilizando seus conhecimentos científicos. Em contrapartida, Galileo nunca quisera trabalhar com isso, tendo se envolvido com o mundo das incertezas a pedido de seu patrono, o grão-duque da Toscana.

Com a Revolução Científica, as fronteiras da aleatoriedade alcançaram a França, através de uma matemático chamado Blaise Pascal. Em Paris, enquanto ainda era jovem, Blaise foi integrado por seu pai a um grupo de discussão conhecido como Académie Mersenne. Nesta sociedade, estavam o filósofo-matemático René Descartes e o grande matemático amador Pierre de Fermat.

Foi assim que pouco tempo depois a Matemática mudaria para sempre, sob a magnífica correspondência formada entre Pascal e Fermat. O trabalho que desenvolveram se mostrou um grande primeiro passo na busca de uma teoria matemática que pudesse descrever a aleatoriedade. 

\end{document}
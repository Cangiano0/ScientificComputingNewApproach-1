\documentclass[a4paper, 12pt]{article}
\usepackage[portuges]{babel}
\usepackage[utf8x]{inputenc}
\usepackage[T1]{fontenc}
\usepackage[a4paper,top=3cm,bottom=2cm,left=3cm,right=3cm,marginparwidth=1.75cm]{geometry} 
\usepackage{float}
\usepackage{graphicx}
\usepackage{caption}
\usepackage{wrapfig}

\usepackage[colorlinks=False]{hyperref}

\usepackage{amsmath, amsfonts, amssymb} 
\usepackage{color} 
\usepackage{enumitem}
\usepackage{minted} 


\begin{document}

\section*{Questão 3 - Sequências em um Baralho}
\large {Um jogo utilizando um baralho comum de 52 cartas é baseado na formação de sequências de razão 1. Nele cada jogador sorteia três cartas observa se uma sequência foi formada e, em seguida, repõem as cartas ao baralho. Assim, pede-se:}
\newline
\newline
\textbf{A.}  A probabilidade de um jogador conseguir uma sequencia na quinta jogada.
\newline
\newline
\textbf{B.} A probabilidade de um jogador conseguir pelo menos uma sequencia até a quinta jogada.
\newline
\newline
\textbf{C.}  A probabilidade de conseguir apenas uma sequencia até a quinta jogada.
\newline
\newline
\textbf{D.} Agora, suponha uma mudança nas regras do jogo e calcule \textbf{A}, \textbf{B} e \textbf{C} para situações em que as 3 cartas sejam reveladas uma a uma e a condição de sucesso seja que elas estejam em ordem crescente.



















\end{document}
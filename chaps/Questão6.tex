\documentclass[a4paper, 12pt]{article}
\usepackage[portuges]{babel}
\usepackage[utf8x]{inputenc}
\usepackage[T1]{fontenc}
\usepackage[a4paper,top=3cm,bottom=2cm,left=3cm,right=3cm,marginparwidth=1.75cm]{geometry} 
\usepackage{float}
\usepackage{graphicx}
\usepackage{caption}
\usepackage{wrapfig}

\usepackage[colorlinks=False]{hyperref}

\usepackage{amsmath, amsfonts, amssymb} 
\usepackage{color} 
\usepackage{enumitem}
\usepackage{minted} 


\begin{document}

\section*{Questão 6 - Teste de HIV}
\large {João e Marcos são calouros de uma universidade e após uma semana de provas, durante um passeio pelo campus, notaram que estava sendo realizada uma campanha pela prevenção do HIV. Observando que estavam sendo realizados testes gratuitos para detecção do vírus no organismo, João logo se interessou.
\newline
Contudo, Marcos logo advertiu o amigo que a chance de um resultado falso positivo era um tanto quanto expressiva, já que ele nunca tinha mantido relações sexuais e o contágio por outras vias deveria ser um tanto quanto raro no caso de João. Assim, supondo que a precisão do teste é 0.995 e que a chance de João ter a doença é 0.0001. Pede-se:}
\newline
\newline
\textbf{A.}  Qual a probabilidade de um resultado falso positivo para o HIV no teste de João?
\newline
\newline
\textbf{B.} Qual a probabilidade de um resultado falso negativo?
\newline
\newline
\textbf{C.}  Supondo que troquemos o teste para um de precisão 0.999. Recalcule os itens \textbf{A} e \textbf{B}.




















\end{document}
\documentclass[a4paper, 12pt]{article}
\usepackage[portuges]{babel}
\usepackage[utf8x]{inputenc}
\usepackage[T1]{fontenc}
\usepackage[a4paper,top=3cm,bottom=2cm,left=3cm,right=3cm,marginparwidth=1.75cm]{geometry} 
\usepackage{float}
\usepackage{graphicx}
\usepackage{caption}
\usepackage{wrapfig}

\usepackage[colorlinks=False]{hyperref}

\usepackage{amsmath, amsfonts, amssymb} 
\usepackage{color} 
\usepackage{enumitem}
\usepackage{minted} 


\begin{document}

\section*{Questão 8 - A Piscina de Bolinhas}
\large {Gabriel é um jovem que adora observar padrões de cores por todos os lugares. Certa vez ele estava em uma festa de aniversário e de longe viu uma piscina de bolinhas. Vendo aquilo e uma caixa de papelão próxima ele selecionou algumas das bolas coloridas e tentou se divertir realizando sorteios sucessivos até o fim da festa e a distribuição do bolo. 
\newline
Dentro dessa caixa foram colocadas 10 bolinhas azuis, 15 amarelas, 17 laranjas, 13 vermelhas, 15 brancas, 20 pretas e 8 verdes, já que após isso ele começou a receber olhares fulminantes de alguns outros convidados. Disso, pergunta-se:}
\newline
\newline
\textbf{A.}  De quantas maneiras distintas pode-se tirar 10 bolinhas da caixa com reposição?
\newline
\newline
\textbf{B.} De quantas maneiras distintas pode-se tirar 10 bolinhas da caixa sem reposição?
\newline
\newline
\textbf{C.}   Qual é a disposição mais provável para cada um dos itens, \textbf{A} e \textbf{B}, e suas probabilidades?
\newline
\newline
\textbf{D.} Recalcule os itens \textbf{A} e \textbf{B} para o caso em que as bolinhas além de coloridas são numeradas.



















\end{document}
\documentclass[a4paper, 12pt]{article}
\usepackage[portuges]{babel}
\usepackage[utf8x]{inputenc}
\usepackage[T1]{fontenc}
\usepackage[a4paper,top=3cm,bottom=2cm,left=3cm,right=3cm,marginparwidth=1.75cm]{geometry} 
\usepackage{float}
\usepackage{graphicx}
\usepackage{caption}
\usepackage{wrapfig}

\usepackage[colorlinks=False]{hyperref}

\usepackage{amsmath, amsfonts, amssymb} 
\usepackage{color} 
\usepackage{enumitem}
\usepackage{minted} 


\begin{document}

\section*{Questão 4 - Métodos Contraceptivos}
\large {Valentina e Enzo são um jovem casal e ao longo do seu tempo de namoro sempre se preocupou com os metódos contraceptivos. Contudo, um dia ao realizar o teste de farmácia, que tem uma porcentagem de falha de 0.005, Valentina espantou-se ao receber um resultado positivo pra gravidez. Com isso, sabendo-se que ela tomava pilulas corretamente, segundo instruções profissionais, e, por isso, a chance de falha do método seja de 0.003. Pede-se:}
\newline
\newline
\textbf{A.}  Calcule a chance de o resultado ser um falso positivo.
\newline
\newline
\textbf{B.} Recalcule o item \textbf{A} adicionando o fator que Enzo utilizou camisinha durante o ato e a chance de falha desse método seja 0.02 e que o uso dos dois métodos sejam eventos independentes.
\newline
\newline
\textbf{C.}  Recalcule \textbf{A} e \textbf{B}, considerando que foi realizado um teste com exame de sangue em que a taxa de erro é 0.0001, no lugar do teste de farmácia.




















\end{document}
%
% Body text font is Palatino!
%

\documentclass[paper=9in:6in,pagesize=pdftex,headinclude=on,footinclude=on,10pt,bibtotoc,pointlessnumbers,normalheadings,DIV=9,twoside=false]{scrbook}

\areaset[0.50in]{4.5in}{8in}

% twoside, openright
\KOMAoptions{DIV=last}

\usepackage{trajan}
 
%\usepackage[brazilian]{babel}
\usepackage[utf8x]{inputenc}
\usepackage[T1]{fontenc}
\usepackage{upgreek}
\usepackage{float}
\usepackage[normal,font={footnotesize,it}]{caption}

\usepackage[sc]{mathpazo}
\linespread{1.05} 

\usepackage{verbatim} % for comments
\usepackage{listings} % for comments


\newcommand{\q}[1]{>>\textit{#1}<<}


\usepackage{tikz}
\usetikzlibrary{arrows,decorations.pathmorphing,backgrounds,positioning,fit,petri}
\usepackage{tkz-graph}
\usepackage[square, numbers, comma, sort&compress]{natbib}  
\usepackage{verbatim}  
\usepackage{tex/vector}  
\usepackage{nomencl}
\usepackage{mathrsfs}
\usepackage{aurical}
\usepackage{calligra}
%\usepackage{tgchorus}
\usepackage{enumitem}
\usepackage{tcolorbox}
%\usepackage{gfsartemisia-euler}
%\usepackage{mathptmx}\\
\usepackage{csvsimple}
%\usepackage{ulem}
\usepackage{bm}
\usepackage{wrapfig}


\newcommand{\av}[1]{\left\langle{#1}\right\rangle}
\newcommand{\bo}[1]{\boldsymbol{#1}}
\newcommand{\nn}{\nonumber \\}
\newcommand{\z}{Z\kern-0.6emZ}
\newcommand{\be}{\begin{equation}}
\newcommand{\ee}{\end{equation}}
\newcommand{\ud}{\mathrm{d}}
\newcommand{\uD}{\mathrm{D}}
\newcommand{\mc}{\mathcal}
%\newcommand{\bm}[1]{\boldsymbol{#1}}
\newcommand{\dd}{\mathrm d}
\newcommand{\half}{\frac{1}{2}}
\newcommand{\intf}{\int_{-\infty}^\infty}
\newcommand{\ch} {\mbox{\Fontauri ch}}
\newcommand{\pa} {\mbox{\Fontauri pa}}
\def\duline#1{\underline{\underline{#1}}}

\newcommand{\addtotoc}[2]{
    \phantomsection
    \addcontentsline{toc}{chapter}{#1}
    #2 \clearpage
}

\usepackage{amsthm}
\usepackage{amsmath,amsfonts,amssymb,amscd,amsthm,xspace}

\theoremstyle{plain}
\newtheorem{example}{Exemplo}[chapter]
\newtheorem{theorem}{Teorema}[chapter]
\newtheorem{corollary}[theorem]{Corol\'ario}
\newtheorem{lemma}[theorem]{Lema}
\newtheorem{proposition}[theorem]{Proposi\c{c}\~ao}
\newtheorem{axiom}[theorem]{Axioma}
\theoremstyle{definition}
\newtheorem{definition}[theorem]{Defini\c{c}\~ao}
\theoremstyle{remark}
\newtheorem{remark}[theorem]{Observa\c{c}\~ao}
\usepackage[centerlast,small,sc]{caption}
\setlength{\captionmargin}{20pt}
\newcommand{\fref}[1]{Figura~\ref{#1}}
\newcommand{\tref}[1]{Tabela~\ref{#1}}
\newcommand{\eref}[1]{Equa\c{c}\~{a}o~\ref{#1}}
\newcommand{\cref}[1]{Cap\'{\i}tulo~\ref{#1}}
\newcommand{\sref}[1]{Se\c{c}\~{a}o~\ref{#1}}
\newcommand{\aref}[1]{Ap\^endice~\ref{#1}}
\newcommand*{\Scale}[2][4]{\scalebox{#1}{$#2$}}


%\newtheorem{theorem}{Theorem}[section]
%\newtheorem{lemma}[theorem]{Lemma}
%\newtheorem{proposition}[theorem]{Proposition}
%\newtheorem{corollary}[theorem]{Corollary}
%\newenvironment{proof}[1][Prova]{\begin{trivlist}
%\item[\hskip \labelsep {\bfseries #1}]}{\end{trivlist}}
%\newenvironment{definition}[1][Definição]{\begin{trivlist}
%\item[\hskip \labelsep {\bfseries #1}]}{\end{trivlist}}
%\newenvironment{example}[1][Exemplo]{\begin{trivlist}
%\item[\hskip \labelsep {\bfseries #1}]}{\end{trivlist}}
%\newenvironment{remark}[1][Observação]{\begin{trivlist}
%\item[\hskip \labelsep {\bfseries #1}]}{\end{trivlist}}

%\newcommand{\qed}{\nobreak \ifvmode \relax \else
 %     \ifdim\lastskip<1.5em \hskip-\lastskip
 %     \hskip1.5em plus0em minus0.5em \fi \nobreak
 %     \vrule height0.75em width0.5em depth0.25em\fi}


\newcommand*\circled[1]{\tikz[baseline=(char.base)]{%
            \node[shape=circle,fill=red!20,draw,inner sep=2pt] (char) {#1};}}



\makenomenclature


\usepackage{makeidx}
\makeindex
\usepackage[totoc]{idxlayout}


\usepackage[ % for hyper links in dvi ending in postscript and PDF
pdftex, % pdf output
pdfborder= 0 0 0,
a4paper, % a4 paper
colorlinks=true, % links are colored
citecolor=green, % color of cite links
linkcolor=cyan, % color of hyperref links
menucolor=blue, % color of Acrobat Reader menu buttons
urlcolor=magenta, % color of page of \url{...}
bookmarksopen=true
]{hyperref}


\usepackage{dsfont}


%% ----------------------------------------------------------------


\title{A New Approach to Scientific Computing}   
\author{Renato Vicente (Org.)} 
\date{\today} 

\begin{document}

\graphicspath{{figs/}}  
%=========================================
\begin{titlepage}
		\centering{
			\textcolor{blue} {\fontsize{16}{16}\selectfont 
			A New Approach to Scientific Computing}
		}\\
			
		\vspace{10mm}
		\centering{\large{Renato Vicente (Org.)}\\
		\small{Dep. de Matem\'atica Aplicada \\
		Instituto de Matemática e Estat\'{\i}stica \\
		Universidade de S\~ao Paulo}}\\
		\vspace{\fill}
		\centering 	\textcolor{blue}{\large{2017}}
\end{titlepage}


%=========================================
\newpage{}
\thispagestyle {empty}

\vspace*{2cm}

\begin{center}
	\Large{\parbox{10cm}{
		\begin{raggedright}
		{\Large 
			\textcolor{violet}{	\textit{O Fortuna velut luna statu variabilis, semper crescis
             aut decrescis.}}
		}
	
		\vspace{.5cm}\hfill{\textcolor{violet}{--- Carmina Burana}}
		\end{raggedright}
	}
}	
\end{center}

\newpage


%=========================================
%\setstretch{1.3}  

%\fancyhead{} 
%\rhead{\thepage} 
%\lhead{}  
%\pagestyle{fancy}  

%\setstretch{1.3}  
%\pagestyle{fancy}  

%\lhead{\emph{Sumário}}  
\tableofcontents  

%% ----------------------------------------------------------------
%\lhead{\emph{Lista de Figuras}}
%\listoffigures  % Write out the List of Figures

%% ----------------------------------------------------------------
%\lhead{\emph{Lista de Tabelas}}  % Set the left side page header to "List of Tables"
%\listoftables  % Write out the List of Tables

%\renewcommand{\nomname}{Lista de Abreviações e Notações}
%\addcontentsline{toc}{chapter}{\nomname}
%\lhead{\emph{Lista de Abreviações}}
%\printnomenclature [5em]

\addtocontents{toc}{\vspace{2em}}  
% PREFACIO

\chapter{\emph{Preface}
\label{preface}








 


\mainmatter	  
%\pagestyle{fancy} 

\setcounter{page}{6}

\chapter{\emph{Introduction}} 
\label{introduction}
 
\chapter{\emph{Probability}} 
\label{probability}

\section{What is probability?}

\section{What is it, really?}

\section{Random variables}

\section{Independence and Bayes theorem}

\section{Probability models}
 
\chapter{\emph{Statistics}} 
\label{statistics}
 

\section{A lady tasting tea}
\section{Essentials of point estimation}
\section{Bootstrapp}
\section{Essentials of Information Theory} 
\chapter{\emph{Difference Equations}} 
\label{differenceeqs}

\section{Finite difference calculus}
\section{Linear first order equations}
\section{Fixed points}
\section{Linear equations of any order}
\section{Systems of linear equations}
\section{Numerical solutions of ODEs}

\chapter{\emph{Solving Equations}} 
\label{solvingequations}


\section{Systems of linear equations}
\section{Equations in $\mathbb{R}$}
\section{Equations in ${\mathbb{R}}^N$}
 
\chapter{\emph{Optimization}} 
\label{optimization}


\section{Linear programming}
\section{Convex problems}
\section{Gradient descent}








 
\chapter{\emph{Training Machines}} 
\label{trainingmachines}

\section{Perceptrons}
\section{Backpropagation}
\section{Stochastic gradients}

 
\chapter{\emph{Spectral Analysis}} 
\label{spectral}


\section{Function approximations}
\section{Fourier spectra}
\section{Orthogonal polynomials}
\section{Neural networks}
\section{Wavelets}
\section{Fast Fourier Transform}

\chapter{\emph{Integration}} 
\label{integration}


\section{Interpolation}
\section{Newtonian quadrature}
\section{Gaussian quadrature}


\chapter{\emph{Inference}} 
\label{inference}


\section{Classic}
\subsection{Least squares}
\subsection{Maximum likelihood}
\subsection{Logistic regression}

\section{Bayesian}
 
\chapter{\emph{Dimensional Reduction}} 
\label{dimensionalreduction}

\section{Eigenproblems}
\section{Principal Component Analysis}
\section{Independent Component Analysis}
\chapter{\emph{Monte Carlo Methods}} 
\label{MCmethods}

\section{Finding $\pi$ }
\section{Ensembles and stochastic processes}
\section{Ising Model}
\section{Metropolis}
\section{Markov Chain Monte Carlo}
\section{Generative models}


\addtocontents{toc}{\vspace{2em}} 
\appendix 

\addtocontents{toc}{\vspace{2em}}  %

\backmatter
\clearpage

\label{Bibliografia}
\bibliographystyle{unsrt}  
\bibliography{Bibliography}   
\printindex

\end{document}
